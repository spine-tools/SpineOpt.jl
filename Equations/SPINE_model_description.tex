%%% !TEX program = lualatex
\documentclass[10pt,english]{article}
\usepackage[a4paper,bindingoffset=0in,left=1.3in,right=1.3in,top=1.2in,bottom=1in,footskip=.25in]{geometry}
\usepackage{lineno,hyperref}
\usepackage[margin=1cm]{caption}
% Packages I added:
%\usepackage{caption,subcaption}
% \usepackage{tablefootnote}
\usepackage{placeins} % For floatbarrier
% \usepackage[utf8]{inputenc}
\usepackage{color}
\usepackage{colortbl}
% \usepackage{arydshln} % For dashed line in table (appendix)
% \usepackage{multirow}
\usepackage{tabularx}
% \newcolumntype{R}{>{\raggedleft\arraybackslash}X}%
% \newcolumntype{C}{>{\centering\arraybackslash}X}%
% \newcolumntype{s}{>{\centering\arraybackslash\hsize=0.8\hsize}X}
% \newcolumntype{z}{>{\centering\arraybackslash\hsize=1.3\hsize}X}
% \newcolumntype{a}{>{\centering\arraybackslash\hsize=0.6\hsize}X}
% \newcolumntype{b}{>{\centering\arraybackslash\hsize=1.1\hsize}X}
% \newcolumntype{y}{>{\centering\arraybackslash\hsize=1.6\hsize}X}
% \newcolumntype{o}{>{\raggedright\arraybackslash\hsize=1.6\hsize}X}
% \newcolumntype{d}{>{\centering\arraybackslash\hsize=0.85\hsize}X}
% \newcolumntype{e}{>{\raggedright\arraybackslash\hsize=0.4\hsize}X}
% \newcolumntype{t}{>{\raggedright\arraybackslash\hsize=1.6\hsize}X}
\usepackage{graphicx}
\usepackage{epstopdf}
\usepackage{amsmath}
\usepackage{amssymb}
\usepackage[caption=false,font=normalsize]{subfig}
%\captionsetup[subfigure]{margin=20pt}
\usepackage{notoccite}
\usepackage{longtable}
% \usepackage{ltxtable}

\usepackage{eurosym}
\usepackage{tikz, pgfplots, graphicx}
\usetikzlibrary{spy,calc}
\usetikzlibrary{positioning}
\pgfplotsset{compat=newest}
\usepackage[space]{grffile} % For defining path variables with spaces
\usepgfplotslibrary{external} 
% \tikzexternalize[prefix=tikz/] % automatically saves pdfs generated by tikz into a folder tikz



%!TEX root = SPINE_model_equations.tex
\newcommand{\pUnitConvCapToFlow}[1][\unit]{p^{UnitConvCapToFlow}_{#1}}
\newcommand{\pImportCost}[1][\commodity,\segment]{p^{ImportCost}_{#1}}
\newcommand{\pConversionCost}[1][\unit,\commodity]{p^{ConversionCost}_{#1}}
\newcommand{\pDeltaT}[1][\timestep]{p^{DeltaT}_{#1}}
\newcommand{\pDemand}[1][\node,\timestep]{p^{Demand}_{#1}}
\newcommand{\pMinimumDownTime}[1][\unit]{p^{MinimumDownTime}_{#1}}
\newcommand{\pMinimumUpTime}[1][\unit]{p^{MinimumUpTime}_{#1}}
\newcommand{\pUnitCapacity}[1][\unit]{p^{UnitCapacity}_{#1}}
\newcommand{\pMinimumOperatingPoint}[1][\unit,cg,\timestep]{p^{MinimumOperatingPoint}_{#1}}
\newcommand{\pProductionShare}[1][\unit,cg,\timestep]{p^{ProductionShare}_{#1}}
\newcommand{\pFlowBound}[1][\unit,cg,\timestep]{p^{FlowBound}_{#1}}
\newcommand{\pRatioOutputOutputFlow}[1][\unit,cg21,cg1]{p^{RatioOutputOutputFlow}_{#1}}
\newcommand{\pRatioInputInputFlow}[1][\unit,cg21,cg1]{p^{RatioInputInputFlow}_{#1}}
\newcommand{\pRatioOutputInputFlow}[1][\unit,cg21,cg1]{p^{RatioOutputInputFlow}_{#1}}
\newcommand{\pConsumptionShare}[1][\unit,cg,\timestep]{p^{ConsumptionShare}_{#1}}
\newcommand{\pMaxStartUpPower}[1][\unit,cg]{p^{MaxStartUpPower}_{#1}}
\newcommand{\ptranstocap}[1][\node_i,\node_j]{p^{TransToCap}_{#1}}
\newcommand{\pRampRateDown}[1][\unit,cg]{p^{RampRateDown}_{#1}}
\newcommand{\pRampRateUp}[1][\unit,cg]{p^{RampRateUp}_{#1}}
\newcommand{\pMaxShutDownPower}[1][\unit,cg]{p^{MaxShutDownPower}_{#1}}
\newcommand{\pAF}[1][\unit,\timestep]{p^{AF}_{#1}}
\newcommand{\commodities}{\commodity \in Commodities}
\newcommand{\ExogenousCommodities}{\commodity \in ExogenousCommodities}
\newcommand{\EndogenousCommodities}{\commodity \in EndogenousCommodities}
\newcommand{\nodes}{\node \in Nodes}
\newcommand{\CommoditiesNodes}{(\commodity, \node) \in CommoditiesNodes}
\newcommand{\ImportSegments}[1][\commodity,\segment]{ImportSegments_{#1}}
\newcommand{\segments}{\segment \in Segments}
\newcommand{\timesteps}{\timestep \in Timesteps}
\newcommand{\vTranCapa}[1][\connection,\node_i,\node_j,\timestep]{v^{TransCapa}_{#1}}
\newcommand{\be}[1][\connection,\timestep]{\beta^{Trans}_{#1}}
\newcommand{\pMaxTransCapa}[1][\connection,\node_i,\node_j]{p^{max.TransCapa}_{#1}}
\newcommand{\ptransloss}[1][\node_i,\node_j]{p^{loss}_{#1}}
\newcommand{\al}[1][\connection,\timestep]{\alpha^{Trans}_{#1}}
\newcommand{\vTrans}[1][\connection,\node_i,\node_j,\timestep]{v^{Trans}_{#1}}
\newcommand{\de}[1][\connection,\node_i,\node_j,\timestep]{\delta^{Trans}_{#1}}
\newcommand{\conbeta}{(\node_i,\node_j,\be)}
\newcommand{\conalpha}{(\node_i,\node_j,\al)}
\newcommand{\nodesconnection}{(\connection,\node_i,\node_j) \in NodepairConnection}
\newcommand{\connections}{\connection \in Connections}
\newcommand{\InputCommoditiesUnitsNodes}{(\commodity, \unit, \node) \in InputCommoditiesUnitsNodes}
\newcommand{\Cdcg}{(\commodity, \unit) \in Cdcg}
\newcommand{\OutputCommoditiesUnits}{(\commodity, \unit) \in OutputCommoditiesUnits}
\newcommand{\InputCommoditiesUnits}{(\commodity, \unit) \in InputCommoditiesUnits}
\newcommand{\OutputCommoditiesUnitsNodes}{(\commodity, \unit, \node) \in OutputCommoditiesUnitsNodes}
\newcommand{\ga}[1][\connection,\node_i,\node_j,\timestep]{\gamma^{Trans}_{#1}}
\newcommand{\units}{\unit \in Units}
\newcommand{\vImportCosts}[1][\commodity,\timestep]{v^{ImportCosts}_{#1}}
\newcommand{\vCapacity}[1][\unit]{v^{Capacity}_{#1}}
\newcommand{\vFlow}[1][\commodity,\node, \unit,in/out,\timestep]{v^{Flow}_{#1}}
\newcommand{\vUnitsShuttingDown}[1][\unit,\timestep]{v^{UnitsShuttingDown}_{#1}}
\newcommand{\vUnitsStartingUp}[1][\unit,\timestep]{v^{UnitsStartingUp}_{#1}}
\newcommand{\vUnitsAvailable}[1][\unit,\timestep]{v^{UnitsAvailable}_{#1}}
\newcommand{\vUnitsOnline}[1][\unit,\timestep]{v^{UnitsOnline}_{#1}}
\newcommand{\vImport}[1][\commodity,\timestep,\segment]{v^{Import}_{#1}}
\newcommand{\unit}{u}
\newcommand{\commodity}{c}
\newcommand{\node}{n}
\newcommand{\commodityendo}{c}
\newcommand{\timestep}{t}
\newcommand{\segment}{s}
\newcommand{\connection}{{con}}



% \bibliographystyle{elsarticle-num}



% \usepackage[colorlinks]{hyperref}
% \usepackage{breakurl}

% correct bad hyphenation here
% \hyphenation{op-tical net-works semi-conduc-tor ESOMs}


% \allowdisplaybreaks

% \bibliographystyle{elsarticle-num}
%%%%%%%%%%%%%%%%%%%%%%%

%\definecolor{LightCyan}{rgb}{0.88,1,1}
\interfootnotelinepenalty=10000

\newcommand{\myparagraph}[1]{\paragraph{#1}\mbox{}\\} % For having a new line after the paragraph title


\hyphenation{Euro-pean ap-pro-xi-ma-ted inte-gra-ting}
\title{Spine Model formulation and design}
\date{}
\author{Kris Poncelet\textsuperscript{*,\#}\\ 
{\small \textsuperscript{\#}dr. ir. Kris Poncelet, Leuven, Belgium, kris.poncelet@mech.kuleuven.be} \\
{\small \textsuperscript{*}University of Leuven (KU Leuven) Energy Institute TME Branch, EnergyVille}}

\begin{document}

\voffset = 0pt

\maketitle{}


\textbf{Included in this document version:}
\begin{itemize}
\item Basic definition of variables, parameters and equations in order to:
\begin{itemize}
\item Establish a set of commodity flow(s) which define the capacity of the unit/technology 
\item A constraint restriciting the commodity flow(s) to the installed capacity of the unit/technology
\item Establisch a linear relationship between different commodity flows entering and leaving a specific unit/technology (in part)
\item Detailed technical constraints (UC type of constraints)
\item A commodity balance constraint
\end{itemize} 
\end{itemize} 


\textbf{Not included:}
\begin{itemize}
\item Geographical regions (treatment of nodes) and network related constraints (Maren, Jody)
\item Storage technologies
\item Temporal model structure (Juha? + Kris)
\item Objective function + flexibility
\item Bounds on absolute or total commodity flows
\item Stochastics (Juha?)
\end{itemize} 

\textbf{Open issues}: 
\begin{itemize}
\item Exogenous commodities: commodity input units or not?
\item What needs to be adapted if different commodity flows are tracked at a different level of temporal granularity?
\item Linking of different equations 'selected" by the archetype (Kris)
\item Feasible to treat reserve capacity as a commodity? -> Different types of reserves and constraints for reserve capacity dependent on type (e.g., spinning, non-spinning, upward, downward, ...) (Kris)
\item Dealing with the ability to have operational and investment planning problems. Many equations contain capacity-related terms. However, depending on the problem, these can be variables or parameters. Need duplicates of equations for both or does Julia/Jump has features to deal with this? (Kris)
\end{itemize}

\textbf{Next steps}:
\begin{itemize}
\item Compare to Backbone implementation + discuss!
\item Experiment with different archetypes for the coal-fired power plant example
\item Introduce the CHP example
\end{itemize}



\clearpage
\tableofcontents


\clearpage
%!TEX root = SPINE_model_equations.tex
%%%%%%%%%%%%%%%%%%%%%%%%%%%%%%%%%%%%%%%%%%%%%%%%%%%%%%%%%%%%%%%%%%%%%%%%%%%
% Nomenclature
%%%%%%%%%%%%%%%%%%%%%%%%%%%%%%%%%%%%%%%%%%%%%%%%%%%%%%%%%%%%%%%%%%%%%%%%%%%
\section*{Nomenclature}
\newcount\totalcol
\totalcol = 3
\newdimen\cola
\cola = 6cm
\newdimen\colb
\colb = 0cm
\newdimen\colc
\colc =\dimexpr\textwidth -\tabcolsep *\totalcol * 2 -\arrayrulewidth * (1 +\totalcol)-\cola -\colb\relax
%%%%%%%%%%%%%%%%%%%%%%%%%%%%%%%%%%%%%%%%%%%%%%%%%%%%%%%%%%%%%%%%%%%%%%%%%%%
% Sets
%%%%%%%%%%%%%%%%%%%%%%%%%%%%%%%%%%%%%%%%%%%%%%%%%%%%%%%%%%%%%%%%%%%%%%%%%%%
\subsection*{Sets}
\vspace{-1em}
	\begin{longtable}{p{\cola} p{\colc} >{\small\raggedleft\arraybackslash\itshape}p{\colb}}
		$\ExogenousCommodities$	& Set of exogenous commodities (i.e., not requiring a balance constraint)	&                \\
		$\EndogenousCommodities$	& Set of endogenous commodities (i.e., requiring a balance constraint)	&                \\
		$\commodities   $	& Set of commodities                                          	&                \\
		$\ImportSegments$	& Set of import segments \segment for commodity \commodity    	&                \\
		$\segments      $	& Set of piecewise linear segments                            	&                \\
		$\timesteps     $	& Set of timesteps                                            	&                \\
		$\InputCommodities$	& Set of units \unit having input commodity  \commodity       	&                \\
		$\OutputCommodities$	& Set of units \unit having output commodity \commodity       	&                \\
		$\units         $	& Set of units/technologies                                   	&                \\
	\end{longtable}

%%%%%%%%%%%%%%%%%%%%%%%%%%%%%%%%%%%%%%%%%%%%%%%%%%%%%%%%%%%%%%%%%%%%%%%%%%%
% Parameters
%%%%%%%%%%%%%%%%%%%%%%%%%%%%%%%%%%%%%%%%%%%%%%%%%%%%%%%%%%%%%%%%%%%%%%%%%%%
\subsection*{Parameters}
\vspace{-1em}
	\begin{longtable}{p{\cola} p{\colc} >{\small\raggedleft\arraybackslash\itshape}p{\colb}}

		$\pUnitConvCapToFlow$	& Converting capacity to flow units of the capacity defining commodity group	&                \\[0.5em]

		$\pImportCost   $	& Cost related to the import of a commodity $\commodity$ within segment $\segment$ 	&                \\[0.5em]

		$\pConversionCost$	& Cost attached to the outflow of commodity $\commodity$ from the unit	&                \\[0.5em]

		$\pDeltaT       $	& Duration of time step t                                     	&                \\[0.5em]

		$\pDemand       $	& Demand for commodity \commodity in timestep \timestep       	&                \\[0.5em]

		$\pMinimumDownTime$	& Minimum time a unit is required to remain offline after shutting down	&                \\
		$\pMinimumUpTime$	& Minimum time a unit is required to remain online after starting up	&                \\
		$\pUnitCapacity $	& Capacity of a single unit                                   	&                \\
		$\pMinimumOperatingPoint$	& Minimum operating point of the unit                         	&                \\
		$\pRatioOutputInputFlow$	& Ratio between output commodity group cg2 and input commodity group cg1	&                \\
		$\pRatioInputInputFlow$	& Ratio between input commodity group cg2 and input commodity group cg1	&                \\
		$\pRatioOutputOutputFlow$	& Ratio between output commodity group cg2 and output commodity group cg1	&                \\
		$\pMaxShutDownPower$	& Maximum power in the last time step before a shutdown       	&                \\
		$\pRampRateUp   $	& Maximum upward ramp rate                                    	&                \\
		$\pRampRateDown $	& Maximum downward ramp rate                                  	&                \\
		$\pMaxStartUpPower$	& Maximum power in first time step after a start-up           	&                \\
		$\pAF           $	& Availability factor                                         	&                \\
	\end{longtable}

%%%%%%%%%%%%%%%%%%%%%%%%%%%%%%%%%%%%%%%%%%%%%%%%%%%%%%%%%%%%%%%%%%%%%%%%%%%
% Variables
%%%%%%%%%%%%%%%%%%%%%%%%%%%%%%%%%%%%%%%%%%%%%%%%%%%%%%%%%%%%%%%%%%%%%%%%%%%
\subsection*{Decision Variables}
\vspace{-1em}
	\begin{longtable}{p{\cola} p{\colc} >{\small\raggedleft\arraybackslash\itshape}p{\colb}}
		$\vImportCosts  $	& Costs related to the import of a commodity in a certain time step	&                \\[0.5em]

		$\vCapacity     $	& Installed capacity of a certain unit                        	&                \\[0.5em]

		$\vFlow         $	& Commodity flow in/out a certain unit in a given time step   	&                \\
		$\vUnitsStartingUp$	& Number of units starting up in time step $\timestep$ (coming online in time step $\timestep +1$)	&                \\
		$\vUnitsShuttingDown$	& Number of units shutting down in time step $\timestep$ (going offline in time step $\timestep +1$)	&                \\
		$\vUnitsOnline  $	& Number of online units                                      	&                \\
		$\vUnitsAvailable$	& Number of available units                                   	&                \\
		$\vImport       $	& Import of commodity $\commodity$ in segment $\segment$      	&                \\
	\end{longtable}









%!TEX root = ../SPINE_model_description.tex


%%%%%%%%%%%%%%%%%%%%%%%%%%%%%%%%%%%%%%%%%%%%%%%%%%%%%
%%%%%%%%%%%%%%%%%%%%%%%%%%%%%%%%%%%%%%%%%%%%%%%%%%%%%
\section{Model formulation}
%%%%%%%%%%%%%%%%%%%%%%%%%%%%%%%%%%%%%%%%%%%%%%%%%%%%%
%%%%%%%%%%%%%%%%%%%%%%%%%%%%%%%%%%%%%%%%%%%%%%%%%%%%%

%%%%%%%%%%%%%%%%%%%%%%%%%%%%%%%%%%%%%%%%%%%%%%%%%%%%%
\subsection{Objective function}
%%%%%%%%%%%%%%%%%%%%%%%%%%%%%%%%%%%%%%%%%%%%%%%%%%%%%

The basic objective function of the Spine Model is to minimize the total discounted costs. 


The total costs will in the end comprise the following elements:
\begin{itemize}
	\item Costs:
	\begin{itemize}
		\item investment costs
		\item dismantling costs
		\item fixed O\&M costs
		\item variable O\&M costs
		\item import costs (exogenous commodity flows with a cost attached)
		\item production costs (related to domestic resource production, e.g., extracting oil from oil fields)
		\item taxes and subsidies associated with commodity flows
		\item taxes and subsidies associated with investments
		\item Utility loss following reduced end-use demands
		\item utility losses related to reduced reliability
	\end{itemize}
	\item Revenues:
	\begin{itemize}
		\item Salvage value (value of investments beyond the considered model horizon)
		\item export revenues
	\end{itemize}
\end{itemize}




\begin{align} \label{eq:objective}
Min_{\vFlow} \quad  TotalCost
\end{align}

\begin{align} \label{eq:objective_definition}
TotalCost = ProductionCost
\end{align}


\begin{align} \label{eq:production_cost}
ProductionCost = \sum_{\timesteps} \Bigg[& \nonumber \\
& \sum_{(\unit, \commodity, \node) \in \OutputCommoditiesUnitsNodes} \Big(\vFlow[\commodity, \node, \unit, out, \timestep] \pConversionCost \Big) \nonumber \\
& \sum_{(\unit, \commodity, \node) \in \InputCommoditiesUnitsNodes} \Big(\vFlow[\commodity, \node, \unit, in, \timestep]\pConversionCost \Big) \nonumber \\
& \Bigg]
%\sum_{\segments} \vImport \pImportCost \quad \forall \commodities, \timesteps
\end{align}
{\color{red} Different segments to add}



{\color{red} TO EXPAND}






%%%%%%%%%%%%%%%%%%%%%%%%%%%%%%%%%%%%%%%%%%%%%%%%%%%%%
\subsection{Technological constraints}
%%%%%%%%%%%%%%%%%%%%%%%%%%%%%%%%%%%%%%%%%%%%%%%%%%%%%

\subsubsection{Define unit/technology capacity}
%%%%%%%%%%%%%%%%%%%%%%%%%%%

In a multi-commodity setting, there can be different commodities entering/leaving a certain technology/unit. These can be energy-related commodities (e.g., electricity, natural gas, etc.), emissions, or other commodities (e.g., water, steel). The capacity of the unit must be unambiguously defined based on certain commodity flows (e.g., the capacity of a CHP could be based on the output electricity, the output heat, the sum of both or the input natural gas). 

\begin{align} \label{eq:max_capacity}
&\sum_{\substack{\commodity :  \\ \commodity \in \commoditygroup \\ \OutputCommoditiesUnits}} \vFlow[\commodity, \node, \unit, out, \timestep] + \sum_{\substack{\commodity :  \\  \commodity \in \commoditygroup \\ \InputCommoditiesUnits}} \vFlow[\commodity, \node, \unit, in, \timestep] \nonumber \\
&  \le \nonumber \\
&\pAF \pUnitCapacity \{\vNumberOfUnits or \quad \pNumberOfUnits\} \pUnitConvCapToFlow \nonumber \\
&\forall \quad \timesteps, (\units,\commoditygroups): \pUnitCapacity \text{ is defined}
\end{align}





% {\color{red} DEPRECATED
% Therefore, a group of commodities need to be defined which define the capacity of the unit/technology. This is the so-called 'Capacity defining commodity group' (cdcg)\footnote{In TIMES, this is called the 'Primary commodity group'}, and should be defined by the users (defaults can apply). This capacity thus restricts the flows of the commodities in the Cdcg (similar to TIMES EQ CAPACT).
% \begin{align} \label{eq:max_capacity}
% &\sum_{\substack{\commodity :  \\ \Cdcg \\ \OutputCommoditiesUnits}} \vFlow[\commodity, \node, \unit, out, \timestep] + \sum_{\substack{\commodity :  \\  \Cdcg \\ \InputCommoditiesUnits}} \vFlow[\commodity, \node, \unit, in, \timestep] \nonumber \\
% &  \le \nonumber \\
% &\pAF \pUnitConvCapToFlow \{\vCapacity or \quad \pUnitCapacity\} \quad \forall \quad \units, \timesteps
% \end{align}
% }



\subsubsection{Static and linear relationships between input and output commodity flows}
%%%%%%%%%%%%%%%%%%%%%%%%%%%
Between the different flows, relationships can be imposed. The most simple relationship is a linear relationship between input and output commodities/commodity groups (TIMES EQ PTRANS). Whenever there is only a single input commodity and a single output commodity, this relationship relates to the notion of an efficiency. This equation can however also be used for instance to relate emissions to input primary fuel flows. In the most general form of the equation, two commodity groups are defined (an input commodity group $cg\_in$ and an output commodity group $cg\_out$)\footnote{The above equations indicate that it might not be so simple as simply defining the value of a number of parameters which either belong to a unit or commodity. The user might also need to specify to which commodity groups different parameters relate.}, and a linear relationship is expressed between both commodity groups. Note that whenever the relationship is specfied between groups of multiple commodities, there remains a degree of freedom regarding the composition of the input commodity flows within group cg1 and the output commodity flows within group cg2. 

\paragraph{Parameter $\pFixRatioOutputInputFlow$: Fixed relationships between output and input flows}
\begin{align} \label{eq:fixratiooutputinputflow}
&\sum_{\commodity \in cg\_out} \vFlow[\commodity,\unit,out,\timestep] = \pFixRatioOutputInputFlow \sum_{\commodity \in cg\_in} \vFlow[\commodity,\unit,in,\timestep] \nonumber \\
&\forall \units, \timesteps, cg\_out, cg\_in : \pFixRatioOutputInputFlow is defined
\end{align}

\paragraph{Parameter $\pMaxRatioOutputInputFlow$: Maximal output flows relative to input flows:}
\begin{align} \label{eq:maxratiooutputinputflow}
&\sum_{\commodity \in cg\_out} \vFlow[\commodity,\unit,out,\timestep] \le \pMaxRatioOutputInputFlow \sum_{\commodity \in cg\_in} \vFlow[\commodity,\unit,in,\timestep] \nonumber \\
&\forall \units, \timesteps, cg\_out, cg\_in : \pMaxRatioOutputInputFlow is defined
\end{align}

\paragraph{Parameter $\pMinRatioOutputInputFlow$: Minimal output flows relative to input flows:}
\begin{align} \label{eq:minratiooutputinputflow}
&\sum_{\commodity \in cg\_out} \vFlow[\commodity,\unit,out,\timestep] \ge \pMinRatioOutputInputFlow \sum_{\commodity \in cg\_in} \vFlow[\commodity,\unit,in,\timestep] \nonumber \\
&\forall \units, \timesteps, cg\_out, cg\_in : \pMinRatioOutputInputFlow is defined
\end{align}

Additional relationships can further be imposed. Two basic constraints impose a linear relationship between multiple input commodities/commodity groups (Eq.~\eqref{eq:fixratioinputinputflow}-\eqref{eq:maxratioinputinputflow} - similar to (TIMES EQ INSHR)), and a linear relationship between multiple output commodities/commodity groups (Eq.~\eqref{eq:fixratiooutputoutputflow}-\eqref{eq:maxratiooutputoutputflow} - similar to TIMES EQ OUTSHR). These relationships reduce the degrees of freedom. The relationship between different input flows can for instance be used to define a fixed or maximal share of biomass in a coal-fired power plant. The relationship between different output flows can for instance be used to define a relationship between the heat and electrical power outputs of a CHP plant, or to establish relationships between different outputs in an distillery. 

\paragraph{Parameter $\pFixRatioInputInputFlow$: Fixed relationship between multiple input flows}
\begin{align} \label{eq:fixratioinputinputflow}
&\sum_{\commodity \in cg2} \vFlow[\commodity,\unit,in,\timestep] = \pFixRatioInputInputFlow \sum_{\commodity \in cg1} \vFlow[\commodity,\unit,in,\timestep] \nonumber \\
&\forall \units, \timesteps, cg1, cg2: \pFixRatioInputInputFlow is defined
\end{align}

\paragraph{Parameter $\pMaxRatioInputInputFlow$: Maximal ratio between multiple input flows}
\begin{align} \label{eq:maxratioinputinputflow}
&\sum_{\commodity \in cg2} \vFlow[\commodity,\unit,in,\timestep] \le \pMaxRatioInputInputFlow \sum_{\commodity \in cg1} \vFlow[\commodity,\unit,in,\timestep] \nonumber \\
&\forall \units, \timesteps, cg1, cg2: \pMaxRatioInputInputFlow is defined
\end{align}

\paragraph{Parameter $\pFixRatioOutputOutputFlow$: Fixed relationship between multiple output flows}
\begin{align} \label{eq:fixratiooutputoutputflow}
&\sum_{\commodity \in cg2} \vFlow[\commodity,\unit,outn,\timestep] = \pFixRatioOutputOutputFlow \sum_{\commodity \in cg1} \vFlow[\commodity,\unit,out,\timestep] \nonumber \\
&\forall \units, \timesteps, cg1, cg2: \pFixRatioOutputOutputFlow is defined
\end{align}

\paragraph{Parameter $\pMaxRatioOutputOutputFlow$: Maximal ratio between multiple output flows}
\begin{align} \label{eq:maxratiooutputoutputflow}
&\sum_{\commodity \in cg2} \vFlow[\commodity,\unit,outn,\timestep] \le \pMaxRatioOutputOutputFlow \sum_{\commodity \in cg1} \vFlow[\commodity,\unit,out,\timestep] \nonumber \\
&\forall \units, \timesteps, cg1, cg2: \pMaxRatioOutputOutputFlow is defined
\end{align}



The above static relationships represent constraints on the ratios between different commodity flows per unit. Additionally, bounds can be put on the instantaneous or total absolute flows generated by each unit, or even on the instantaneous or total flows from all units together (the latter are no longer technological constraints though).

\subsubsection{Absolute bounds on commodity flows}
%%%%%%%%%%%%%%%%%%%%%%%%%%%

Different absolute bounds on the flows of a commodity/the commodities within a commodity group can be generated. These bounds can be on the total production (output), consumption or on the net production of a certain commodities. In addition, the bounds can be generated for different dimensions:
\begin{enumerate}
	\item for a specific unit at a certain moment
	\item for a specific unit over all time steps
	\item for a specific unit, cumulative over different years
	\item for all units within a certain region at a certain moment
	\item for all units within a certain region over all time steps
	\item for all units within a certain region, cumulative over different years
	\item for all units (accross all regions) at a certain moment
	\item for all units (accross all regions) over all time steps
	\item for all units (across all regions), cumulative over different years
\end{enumerate}
Each combination would possible require a different parameter and equation (here only for absolute bounds already 27 in total). This could be strongly reduced by using unit groups and commodity groups (preferably facilitated by having default groups such as for all units, for all timesteps, etc.). {\color{red} Temporal emements still need to be included. I think that the bounds could be specified for timestep groups within a year. For multi-year models, cumulative bounds over different years could be imposed as well.}


 The generic equations constraining the instantaneous production, consumption and net production of commodities respectively then look as follows: 

\paragraph{Parameters $\pFixOutFlowBound$, $\pMaxOutFlowBound$, $\pMinOutFlowBound$: Bound on the instantaneous production:}
\begin{align} \label{eq:outflowbound}
&\sum_{\substack{\commodity \in \commoditygroup, \unit \in \unitgroup, \node \in \nodegroup: \\ \OutputCommoditiesUnitsNodes}} \vFlow[\commodity,\node,\unit,out,\timestep]  \{=;\le;\ge\} \{\pFixOutFlowBound;\pMaxOutFlowBound;\pMinOutFlowBound\} \nonumber \\
&\forall \commoditygroups, \unitgroups, \timesteps : \nonumber \\
&\{\pFixOutFlowBound;\pMaxOutFlowBound;\pMinOutFlowBound\} specified
\end{align}

\paragraph{Parameters $\pFixInFlowBound$, $\pMaxInFlowBound$, $\pMinInFlowBound$: Bound on the instantaneous consumption:}
\begin{align} \label{eq:inflowbound}
&\sum_{\substack{\commodity \in \commoditygroup, \unit \in \unitgroup, \node \in \nodegroup: \\ \InputCommoditiesUnitsNodes}} \vFlow[\commodity,\node,\unit,in,\timestep]  \{=;\le;\ge\} \{\pFixInFlowBound ; \pMaxInFlowBound ; \pMinInFlowBound\} \nonumber \\
&\forall \commoditygroups, \unitgroups, \timesteps : \{\pFixInFlowBound ; \pMaxInFlowBound ; \pMinInFlowBound\} specified
\end{align}

\paragraph{Parameters $\pFixNetFlowBound$, $\pMaxNetFlowBound$, $\pMinNetFlowBound$: Bound on the instantaneous net production:}
\begin{align} \label{eq:netflowbound}
&\sum_{\substack{\commodity \in \commoditygroup, \unit \in \unitgroup, \node \in \nodegroup: \\ \OutputCommoditiesUnitsNodes}} \vFlow[\commodity,\node,\unit,out,\timestep] - \sum_{\substack{\commodity \in \commoditygroup, \unit \in \unitgroup, \node \in \nodegroup: \\ \InputCommoditiesUnitsNodes}} \vFlow[\commodity,\node,\unit,in,\timestep]  \nonumber \\
&\{=;\le;\ge\} \{\pFixNetFlowBound;\pMaxNetFlowBound;\pMinNetFlowBound\} \nonumber \\
&\forall \commoditygroups, \unitgroups, \timesteps : \{\pFixNetFlowBound;\pMaxNetFlowBound;\pMinNetFlowBound \} specified
\end{align}

 The generic equations constraining the cumulative production, consumption and net production of commodities respectively then look as follows {\color{red} To be refined once the temporal representation is fixed}: 

\paragraph{Parameters $\pFixCumOutFlowBound$, $\pMaxCumOutFlowBound$, $\pMinCumOutFlowBound$: Bound on the cumulative production:}
\begin{align} \label{eq:cumoutflowbound}
&\sum_{\timesteps} \sum_{\substack{\commodity \in \commoditygroup, \unit \in \unitgroup, \node \in \nodegroup: \\ \OutputCommoditiesUnitsNodes}} \vFlow[\commodity,\node,\unit,out,\timestep]  \{=;\le;\ge\} \{\pFixCumOutFlowBound ; \pMaxCumOutFlowBound , \pMinCumOutFlowBound \} \nonumber \\
&\forall \commoditygroups, \unitgroups: \{\pFixCumOutFlowBound ; \pMaxCumOutFlowBound , \pMinCumOutFlowBound \} specified
\end{align}

\paragraph{Parameters $\pFixCumInFlowBound$, $\pMaxCumInFlowBound$, $\pMinCumInFlowBound$: Bound on the cumulative consumption:}
\begin{align} \label{eq:cuminflowbound}
&\sum_{\timesteps} \sum_{\substack{\commodity \in \commoditygroup, \unit \in \unitgroup, \node \in \nodegroup: \\ \InputCommoditiesUnitsNodes}} \vFlow[\commodity,\node,\unit,in,\timestep]  \{=;\le;\ge\} \{\pFixCumInFlowBound ; \pMaxCumInFlowBound ; \pMinCumInFlowBound \} \nonumber \\
&\forall \commoditygroups, \unitgroups : \{\pFixCumInFlowBound ; \pMaxCumInFlowBound ; \pMinCumInFlowBound \} specified
\end{align}

\paragraph{Parameters $\pFixCumNetFlowBound$, $\pMaxCumNetFlowBound$, $\pMinCumNetFlowBound$: Bound on the cumulative net production:}
\begin{align} \label{eq:cumnetflowbound}
&\sum_{\timesteps} \sum_{\substack{\commodity \in \commoditygroup, \unit \in \unitgroup, \node \in \nodegroup: \\ \OutputCommoditiesUnitsNodes}} \vFlow[\commodity,\node,\unit,out,\timestep] - \sum_{\timesteps} \sum_{\substack{\commodity \in \commoditygroup, \unit \in \unitgroup, \node \in \nodegroup: \\ \InputCommoditiesUnitsNodes}} \vFlow[\commodity,\node,\unit,in,\timestep]  \nonumber \\
&\{=;\le;\ge\} \{\pFixCumNetFlowBound; \pMaxCumNetFlowBound; \pMinCumNetFlowBound \} \nonumber \\
&\forall \commoditygroups, \unitgroups: \{\pFixCumNetFlowBound; \pMaxCumNetFlowBound; \pMinCumNetFlowBound \} specified
\end{align}



\subsubsection{Relative bounds on commodity flows}
%%%%%%%%%%%%%%%%%%%%%%%%%%%

\paragraph{Parameters $\pFixProductionShare$, $\pMaxProductionShare$, $\pMinProductionShare$: Bound on the instantaneous production share:}
(Similar to EQ FLMRK in TIMES)
\begin{align} \label{eq:productionshare}
&\sum_{\commodity \in \commoditygroup} \sum_{\substack{\unit: \\ \unit \in \unitgroup \\ \OutputCommoditiesUnits}} \vFlow[\commodity,\node,\unit,out,\timestep]
 \{=;\le;\ge\} \nonumber \\
&\{\pFixProductionShare ; \pMaxProductionShare ; \pMinProductionShare \} \sum_{\commodity \in \commoditygroup} \sum_{\substack{\units: \\ \OutputCommoditiesUnits}} \vFlow[\commodity,\node, \unit,out,\timestep] \nonumber \\
&\quad \forall \commoditygroups, \unitgroups, \timesteps : \{\pFixProductionShare ; \pMaxProductionShare ; \pMinProductionShare \} defined
\end{align}

\paragraph{Parameters $\pFixConsumptionShare$, $\pMaxConsumptionShare$, $\pMinConsumptionShare$: Bound on the instantaneous consumption share:}
(Similar to EQ FLMRK in TIMES)
\begin{align} \label{eq:consumptionshare}
&\sum_{\commodity \in \commoditygroup} \sum_{\substack{\unit: \\ \unit \in \unitgroup \\ \InputCommoditiesUnits}} \vFlow[\commodity,\node,\unit,in,\timestep]
 \{=;\le;\ge\} \nonumber \\
&\{ \pFixConsumptionShare ; \pMaxConsumptionShare ; \pMinConsumptionShare \} \sum_{\commodity \in \commoditygroup} \sum_{\substack{\units: \\ \InputCommoditiesUnits}} \vFlow[\commodity,\node, \unit,in,\timestep] \nonumber \\
&\quad \forall \commoditygroups, \unitgroups, \timesteps : \{ \pFixConsumptionShare ; \pMaxConsumptionShare ; \pMinConsumptionShare \} defined
\end{align}


{\color{red}From here: cumulative bounds - to be refined when temporal structure is fixed!}
\paragraph{Parameters $\pFixCumProductionShare$, $\pMaxCumProductionShare$, $\pMinCumProductionShare$: Bound on the cumulative production share:}
\begin{align} \label{eq:cumproductionshare}
&\sum_{\timesteps} \sum_{\commodity \in \commoditygroup} \sum_{\substack{\unit: \\ \unit \in \unitgroup \\ \OutputCommoditiesUnits}} \vFlow[\commodity,\node,\unit,out,\timestep]  \{=;\le;\ge\}  \nonumber \\
&\{ \pFixCumProductionShare ; \pMaxCumProductionShare ; \pMinCumProductionShare \} \sum_{\timesteps} \sum_{\commodity \in \commoditygroup} \sum_{\substack{\units: \\ \OutputCommoditiesUnits}} \vFlow[\commodity,\node, \unit,out,\timestep] \nonumber \\
&\quad \forall \commoditygroups, \unitgroups: \{ \pFixCumProductionShare ; \pMaxCumProductionShare ; \pMinCumProductionShare \} defined
\end{align}

\paragraph{Parameters $\pFixCumConsumptionShare$, $\pMaxCumConsumptionShare$, $\pMinCumConsumptionShare$: Bound on the cumulative consumption share:}
\begin{align} \label{eq:cumconsumptionshare}
&\sum_{\timesteps} \sum_{\commodity \in \commoditygroup} \sum_{\substack{\unit: \\ \unit \in \unitgroup \\ \InputCommoditiesUnits}} \vFlow[\commodity,\node,\unit,in,\timestep] \{=;\le;\ge\} \nonumber \\
&\{ \pFixCumConsumptionShare ; \pMaxCumConsumptionShare ; \pMinCumConsumptionShare \} \sum_{\timesteps} \sum_{\commodity \in \commoditygroup} \sum_{\substack{\units: \\ \InputCommoditiesUnits}} \vFlow[\commodity,\node, \unit,in,\timestep] \nonumber \\
&\quad \forall \commoditygroups, \unitgroups: \{ \pFixCumConsumptionShare ; \pMaxCumConsumptionShare ; \pMinCumConsumptionShare \} defined
\end{align}


Note that for the commodities correspondong to the units' capacity defining commodity group, a bound on the commodity flows is already generated (restricting flows to the installed capacity) - see Eq.~\eqref{eq:max_capacity} or Eq.~\eqref{eq:maximumoperatingpoint}.

{\color{red} TO ELABORATE

Equations to Check:
\begin{itemize}
	\item EQ ACTBND: relates to bound of the flow of a process for a certain time-slice level which is below or above the time-slice level of the process -> Remains to be seen how the temporal dimension will look like in SPINE
	\item EQ BLND
	\item EQ BNDPRD/NET - Bound on the cumulative net bound of a commodity over a time interval
	\item EQ CUMNET/CUMPRD - to generate a bound on the total cumulative production/net production of a commodity over multiple years - similar to Eq.~\eqref{eq:flowbound}, but cumulative over multiple years -> to do when temporal structure is defined
	\item EQ CUMPRD - used to define the total supply of a commodity in each period and time slice (from all different sources).
	\item EQ FLMRK - market share of a unit in the total production of a commodity (e.g., usefull for reserves)
	\item EQ FLOBND - bound on the absolute flows within a commodity group for a specific unit
	\item EQ FLOFR - used to define load curves in TIMES - I guess this will be done differently in Spine
\end{itemize}
}


{\color{red}
TO DO:
\begin{itemize}
	\item adapt flow relationship or bound equations to changing variable definition
	\item generalize equations to be defined for unit groups rather than individual units
	\item include different types of units (regular, storage, network flows) into the flow relationship or bound equations
	\item the current equations should be represented on different temporal levels (e.g., instantaneous, on annual level, over all years, etc.)
	\item Revisit EQ CUMNET/CUMPRD when the temporal structure is defined.
\end{itemize}
}

(similar to TIMES EQ BNDNET/BNDPRD)





\subsubsection{Dynamic constraints on input and output commodity flows}
%%%%%%%%%%%%%%%%%%%%%%%%%%%
\paragraph{Ramping constraints} These constraints induce a bound on the rate of change of a flow of certain commodities/commodity groups. The commodity group cg to which the ramping constraint applies needs to be specified.

There are many different possible formulations of ramping constraints. Hence, the equation is dependent on the archetype selected. Below is are two ramping equation versions represented: one for archetypes which do not have commitment variables, and one for archetypes which do have commitment variables

Without commitment variables ({\color{red} Should in principle be based on available rather than total capacity}):
\begin{align} \label{eq:updwardrampingconstraintwithoutcommitmentvariables}
&\sum_{\commodity \in \commoditygroup} \Big( \vFlow[\commodity,\unit,in/out,\timestep+1] - \vFlow \Big) \le \pRampRateUp \pUnitCapacity \{\vNumberOfUnits or \quad \pNumberOfUnits\} \pDeltaT \nonumber \\
&\forall \units, \timesteps
\end{align}

\begin{align} \label{eq:downdwardrampingconstraintwithoutcommitmentvariables}
&\sum_{\commodity \in \commoditygroup} \Big( \vFlow[\commodity,\unit,in/out,\timestep+1] - \vFlow \Big) \le \pRampRateDown \pUnitCapacity \{\vNumberOfUnits or \quad \pNumberOfUnits\}\pDeltaT \nonumber \\
& \forall \units, \timesteps
\end{align}



With commitment variables:
\begin{align} \label{eq:updwardrampingconstraintwithcommitmentvariables}
\sum_{\commodity \in cg} \Big( \vFlow[\commodity,\unit,in/out,\timestep+1] - \vFlow \Big) \le & (\vUnitsOnline-\vUnitsShuttingDown) \pRampRateUp \pUnitCapacity \pDeltaT \nonumber \\
& - \vUnitsShuttingDown \pMinimumOperatingPoint \nonumber \\
& +\vUnitsStartingUp \pMaxStartUpPower \nonumber \\
& \forall \units, \timesteps
\end{align}

\begin{align} \label{eq:downdwardrampingconstraintwithcommitmentvariables}
\sum_{\commodity \in cg} \Big( \vFlow - \vFlow[\commodity,\unit,in/out,\timestep+1] \Big) \le &(\vUnitsOnline-\vUnitsShuttingDown) \pRampRateDown \pUnitCapacity \pDeltaT \nonumber \\
& - \vUnitsStartingUp \pMinimumOperatingPoint \nonumber \\
& + \vUnitsShuttingDown \pMaxShutDownPower \nonumber \\
& \forall \units, \timesteps
\end{align}






\subsubsection{Commitment-related constraints}
%%%%%%%%%%%%%%%%%%%%%%%%%%%
For modeling certain technologies/units, it is important to not only have flow variables of different commodities, but also model the on/off ("commitment") status of the unit/technology at every time step. Therefore, an additional variable $\vUnitsOnline$ is introduced. This variable represents the number of online units of that technology (for a normal unit commitment model, this variable might be a binary, for investment planning purposes, this might also be an integer or even a continuous variable - this will depend on the archetype of the unit.)

Commitment variables will be introduced by the following constraints (with corresponding parameters):
\begin{itemize}
	\item Minimum operating point ($\pMinimumOperatingPoint$)
	\item Minimum up time ($\pMinimumUpTime$)
	\item Minimum down time ($\pMinimumDownTime$)
	\item Certain ramp-rate formulations depending on the archetype ($\pRampRateUp$, $\pRampRateDown$)
\end{itemize}

Additionally, start-up and shut-down variables might need to be introduced for modeling start-up costs, minimum up time and minimum down-time constraints.

Whenever commitment variables are introduced, the capacity constraint (Eq.~\eqref{eq:max_capacity}) needs to be redefined:
\begin{align} \label{eq:maximumoperatingpoint}
&\sum_{\commodity : \commodity \in \commoditygroup} \vFlow \le \vUnitsOnline \pUnitCapacity \nonumber \\
&\forall \quad \timesteps, (\units, \commoditygroups): \pUnitCapacity \text{ is defined}
\end{align}

Additionally, the number of online units need to be restricted to the installed and available capacity:
\begin{equation} \label{eq:maximumonlineunits}
\vUnitsOnline \le \vUnitsAvailable \quad \forall \quad \units, \timesteps
\end{equation}
\begin{equation} \label{eq:availableunits}
\vUnitsAvailable \le \pAF \{\vNumberOfUnits or \quad \pNumberOfUnits\} \quad \forall \quad \units, \timesteps
\end{equation}

\paragraph{Minimum operating point}
A first commitment-related constraint is the minimal operating point of an online unit. The minimum operating point can be based on the flows of input or output commodities/commodity groups cg ({\color{red}
Is this always for the capacity defining commodity group, or are there instances where a minimum operating point is defined for other commodities/commodity groups?} See example below, if reserve capacity and electrical power together form the Cdcg of a coal-fired power plant, than the Cdcg should not be used here):
\begin{align} \label{eq:minimumoperatingpoint}
&\sum_{\commodity \in cg} \vFlow \ge \pMinimumOperatingPoint \vUnitsOnline \pUnitCapacity \nonumber \\
& \forall \units, \timesteps
\end{align}

{\color{red} To check: how to approach the installed capacity - this can be a parameter or a variable (or both) dependending on the problem?}


\paragraph{Minimum up time}
\begin{align} \label{eq:minimumuptime}
&\vUnitsShuttingDown \le \vUnitsOnline - \sum_{\timestep'=1}^{\pMinimumUpTime - 1} \vUnitsStartingUp[\unit,\timestep-\timestep'] \nonumber \\
& \forall \units, \timesteps
\end{align}
{\color{red} This is the basic constraint. However, whenever non-spinning downward reserves are considered, an additional term which represents 'the units available to shut down in order to provide downward reserves' needs to be added to the left-hand side of the equation. How to deal with this in a generic way?}

\paragraph{Minimum down time}
\begin{align} \label{eq:minimumdowntime}
&\vUnitsStartingUp \le \vUnitsAvailable - \vUnitsOnline - \sum_{\timestep'=1}^{\pMinimumDownTime - 1} \vUnitsShuttingDown[\unit,\timestep-\timestep'] \nonumber \\
& \forall \units, \timesteps
\end{align}
{\color{red} This is the basic constraint. However, whenever non-spinning upward reserves are considered, an additional term which represents 'the units available to start up in order to provide upward reserves' needs to be added to the left-hand side of the equation. How to deal with this in a generic way?}






%%%%%%%%%%%%%%%%%%%%%%%%%%%%%%%%%%%%%%%%%%%%%%%%%%%%%
\subsection{System constraints}
%%%%%%%%%%%%%%%%%%%%%%%%%%%%%%%%%%%%%%%%%%%%%%%%%%%%%

For each endogenous commodity, a commodity balance constraint is induced. The user is free to define whether an inequality or equality sign is used for the balance - TIMES EQ COMBAL
\begin{align} \label{eq:commodity_balance}
&\sum_{\unit : (\commodity,\unit) \in \OutputCommoditiesUnits} \vFlow[\commodity,\node,\unit,out,\timestep] \{\ge;=\} \nonumber \\ 
&\pDemand + \sum_{\unit: (\commodity\unit) \in \InputCommoditiesUnits} \vFlow[\commodity,\node,\unit,in,\timestep] \nonumber \\ 
& \forall \EndogenousCommodities, \timesteps 
\end{align}




%!TEX root = ../SPINE_model_description.tex

\clearpage
%%%%%%%%%%%%%%%%%%%%%%%%%%%%%%%%%%%%%%%%%%%%%%%%%%%%%
%%%%%%%%%%%%%%%%%%%%%%%%%%%%%%%%%%%%%%%%%%%%%%%%%%%%%
\section{Open issues}
%%%%%%%%%%%%%%%%%%%%%%%%%%%%%%%%%%%%%%%%%%%%%%%%%%%%%
%%%%%%%%%%%%%%%%%%%%%%%%%%%%%%%%%%%%%%%%%%%%%%%%%%%%%

%!TEX root = ../../SPINE_model_description.tex


\subsection{On treatment of exogenous commodities} \label{subsec:exogenous_commodities}



\paragraph{Description of the issue:}
Certain commodities are only used as inputs in the model but there are no conversion process creating these commodities in the model. These are called exogenous commodities. Regarding the interpretation, exogenous commodities can be assumed to be imported or produced by certain processes which are not considered explicitly in the model. Such commodities typically have a cost or emissions attached to their use, but are not constrained typically. The issue here is how to deal with such commodities (or in the first place, whether there should be a separate treatment for such commodities in the first place).

Below, two options are presented. One where exogenous commodities are treated differently, and one where exogenous commodities are treated identical to other commodities.  In the description below, we assumed that the costs related to importing a single unit of certain commodity can either be constant but can also vary with the total imported amount. 

\paragraph{Option 1: Separate treatment of exogenous commodities}
Import variables ($\vImport$) are created (one variable per commodity, timestep on which the commodity is traced, and segment in the piecewise linear cost curve). An equation is added to ensure that the total import (sum of the different import segments) equals the net flow of that commodity. No balance constraint on the flow variables of this commodity should be induced (the constraint below replaces this). 
\begin{align}
&\sum_{\segment \in \ImportSegments} \vImport =  \sum_{\unit \in \InputCommodities} \vFlow[\commodity,\unit,in,\timestep]  - \sum_{\unit \in \OutputCommodities} \vFlow[\commodity,\unit,out,\timestep]  \nonumber \\
& \forall \commodities, \timesteps
\end{align}

\paragraph{Option 2: No separate treatment of exogenous commodities}
An 'import unit' is created which has no input commodities but has the specific commoditiy as an output commodity. A cost is attached to the 'generation' of the commodity. In this option, there must be a commodity balance (Eq.~\eqref{eq:commodity_balance}) equation to make sure that the output flows of the import unit are sufficient (and hence, the correct cost can be attached). In addition, when piecewise segments are being used (non-constant import costs), additional import variables ($\vImport$) need to be created and the sum of the different segments should be equated to the output flow of that process
\begin{align}
&\sum_{\segment \in \ImportSegments} \vImport =  \vFlow[\commodity,ImportUnit,out,\timestep] \nonumber \\
& \forall \commodities, \timesteps
\end{align}

Drawbacks of option 2:
\begin{itemize} 
	\item In comparison to option 1, there is one more constraint per commodity, and timestep on which the commodity is traced.
	\item In comparison to option 1, there is an additional variable per commodity and timestep ($\vFlow[\commodity,ImportUnit,out,\timestep]$)
\end{itemize}
Advantages of option 2:
\begin{itemize}
	\item There is a flow variable related to the import of that commodity -> one variable $\vFlow$ now describes all the flows (whereas if you want to say something about the 'import flows' in option 1, you would need to go look at the $\vImport$ variables). This can be easyer for post-processing.
	\item Using a unit for importing might provide more flexibility as all the generic unit parameters are available to model specific constraints related to the imports. For instance, a cost can be placed on the capacity for importing, etc. 
\end{itemize}


Note that all commodities defined in the data will be treated explicitly (so also the exogenous commodities)! For example, in a UC model, there will be flow variables for, for instance, coal consumed by a coal-fired power plant. This might not be efficient for all cases. If the user wants to overcome this issue, the user is free to not define the coal commodity in the model but rather define a generation cost of the coal-fired power plant (parameter $\pConversionCost$).\footnote{I don't see any other option. Ofcourse, it should be possible to start from a datasource where fuel commodities are specified, then make a simple conversion tool to adapt the parameters of the plants involved based on this price, and subsequently run the Spine model.} 


%!TEX root = ../../SPINE_model_description.tex

\subsection{Treatment of reserve commodities} \label{subsec:reserve_commodities}

\paragraph{Description of the issue:} For generality, reserves can be modeled as commodities which are not different from any other commodity. However, the provision of reserves is constrained by highly specific constraints, which are different from most other commodities. Fitting the reserve-related constraints to a generic format might complicate things making the model less transparent and user friendly. Another option would therefore be to have commodity attributes to indicate whether a certain commodity is a reserve commodity (and even more so, whether it is an upward/downward and spinning/non-spinning reserve commodity). While this is easier from a user perspective, it also implies that the terminology is not generic, and that there can/will be optional and non-generic terms in some of the equations.


\paragraph{Option 1: Reserve commodities as a generic commodity}
Reserve-related variables ($\vFlow[ReserveCommodity,\unit,out,\timestep]$) should be considered in many of the very basic constraints. This makes that these correctly defining these basic constraints can become quite complex. 

To illustrate the complexity, consider the case where downward spinning reserves are considered. 

Whenever spinning downward reserves are procured, this impacts the instanteous electricity generation via the minimum operating point constraint (Eq.~\eqref{eq:minimumoperatingpoint}), i.e., the equation should read:
\begin{align} 
& \vFlow[Electricity,\unit,out,\timestep] - \vFlow[DownSpinRes,\unit,out,\timestep] \ge \pMinimumOperatingPoint \vUnitsOnline \pUnitCapacity \nonumber \\
& \forall \units, \timesteps
\end{align}
The general minimum operating point equation (Eq.~\eqref{eq:minimumoperatingpoint}) does currently not support this modification\footnote{It does support the summation over multiple commodity flows, but the flow variable of the downward spinning reserves commodity has a negative sign which is not supported).}. I guess there will always be a way to define a generic way of introducing a generic constraint which allows to generically implement this constraint (and it should be able to do so even with the user constraints). However, from a user perspective, this might be very cumbersome... 



\paragraph{Option 2: reserve commodities as a special type of commodity}
Whenever commodities are defined as being upward/downward and spinning/non-spinning reserve commodities, the model can directly integrate the variables in the required equations. The user should not worry about this and only needs to identify which commodities are reserve commodities. 

For instance, the minimum operating point constraint (Eq.~\eqref{eq:minimumoperatingpoint}) can be modified to be as follows:
\begin{align}
&\sum_{\commodity \in cg} \vFlow - \sum_{\substack{\commodity :  \\
													\commodity \in DownSpinResCommodities \\
													\commodity \in \OutputCommodities}}
										&\vFlow \nonumber \\
										&\ge \nonumber \\
& \pMinimumOperatingPoint \vUnitsOnline \pUnitCapacity \nonumber \\
& \forall \units, \timesteps
\end{align}

The user should in this case thus simply define the parameter $\pMinimumOperatingPoint$ for the commodity group cg being electricity in this case, and define the reserve commodity as a downward spinning reserve commodity.

To sum up,
Advantages Option 2:
\begin{itemize}
\item Defining the equations is easy once it is known whether a commodity is a spinning/non-spinning upward/downward reserve commodity.
\item The task for the user is restricted. The user only needs to indicate whether a commodity is a reserve commodity (and what type of reserve), and provide the basic plant parameters and the reserve-related variables will appear correctly in all constraints. In contrast, in Option 1, the constraints will have to be defined so generically that the user will have to carefully handle for instance the commodity groups and coefficients for each commodity in the commodity group to which the constraint applies.
\end{itemize}
Disadvantages Option 2:
\begin{itemize}
\item Non-generic terms appearing in generic constraints
\item Certain constraints can only be applied to the flow of reserve commodities (e.g., ramping constraint restricting the provision of reserves. In contrast to the regular ramping constraint, this ramping constraint takes place within a single time step.)
\end{itemize}

My preference goes to option 2.


%!TEX root = ../SPINE_model_description.tex


\clearpage
%%%%%%%%%%%%%%%%%%%%%%%%%%%%%%%%%%%%%%%%%%%%%%%%%%%%%
%%%%%%%%%%%%%%%%%%%%%%%%%%%%%%%%%%%%%%%%%%%%%%%%%%%%%
\section{Examples}
%%%%%%%%%%%%%%%%%%%%%%%%%%%%%%%%%%%%%%%%%%%%%%%%%%%%%
%%%%%%%%%%%%%%%%%%%%%%%%%%%%%%%%%%%%%%%%%%%%%%%%%%%%%

%%%%%%%%%%%%%%%%%%%%%%%%%%%%%%%%%%%%%%%%%%%%%%%%%%%%%
\subsection{Thermal power plant}
%%%%%%%%%%%%%%%%%%%%%%%%%%%%%%%%%%%%%%%%%%%%%%%%%%%%%
As a first example, let's consider the DA scheduling of a simple dispatchable coal power plant. In this example, the power plant is assumed to have a single input commodity, namely coal, and three output commodities: electricity, upward spinning reserve capacity and greenhouse gas emissions. The operations of the power plant are assumed to be characterized by the following equations:
\begin{itemize}
\item maximum power + availability factor 
\item fuel consumption
\item greenhouse gas emissions
\item minimum operating point
\item minimum up time
\item minimum down time
\item ramp rate restrictions for energy provision
\item ramp rate restrictions for reserve capacity provision
\end{itemize}

The following sections elaborate on how each of these constraints is/can be implemented in the generic model formulation.

\subsubsection{Maximum power}
%%%%%%%%%%%%%%%%%%%%%%%%%%%
The output of the coal-fired power plant (power and upward reserve capacity) is restricted by Eq.~\eqref{eq:max_capacity}. For this unit, the capacity defining commodity group (Cdcg) consists of the commodities electrical power and reserve capacity. Furthermore, both capacity and flow variables are in the same units ([MW]) and hence the parameter $\pUnitConvCapToFlow$ equals 1. The equations hence becomes:
\begin{align}
\vFlow[ElectricalPower,u,out,t] + \vFlow[UpwardReserves,u,out,t] \le \pAF \vCapacity \quad \forall \units, \timesteps \nonumber
\end{align}

\subsubsection{Fuel consumption}
%%%%%%%%%%%%%%%%%%%%%%%%%%%
The fuel consumption can be defined by establishing a relationship between the flow variable for electrical power, and the flow variable for coal. Here, a linear relationship between these flows is assumed (corresponding to a constant efficiency). This relationship can be imposed by Eq.~\eqref{eq:ratiooutputinputflow}. Here, cg2 refers to the electrical power and cg1 refers to coal. The parameter $\pRatioOutputInputFlow[u,ElectricalPower,Coal]$ thus corresponds to the efficiency $\eta$ of the coal-fired power plant. The equation becomes
\begin{align}
\vFlow[ElectricalPower,u,out,t] = \eta \vFlow[Coal,u,in,t] \quad \forall \units, \timesteps
\end{align}

\subsubsection{GHG emissions}
%%%%%%%%%%%%%%%%%%%%%%%%%%%
The emissions of greenhouse gases can be defined by establishing a relationship between the flow variable for greenhouse gas emissions, and the flow variable for coal\footnote{Note that an alternative option would be to establish a relationship between the flow variable of greenhouse gas emissions and the flow variable for electrical power (via Eq.~\eqref{eq:ratiooutputoutputflow}).}. This relationship can again be imposed by Eq.~\eqref{eq:ratiooutputinputflow}. Here, cg2 refers to the greenhouse gas emissions and cg1 refers to coal. The parameter $\pRatioOutputInputFlow[u,GreenhouseGas,Coal]$ thus corresponds to the emission factor/emission intensity $EF$ of the coal commodity. 
\begin{align}
\vFlow[GreenhouseGas,u,out,t] =  EF \vFlow[Coal,u,in,t] \quad \forall \units, \timesteps
\end{align}

{\color{red} An alternative option would be to link emissions directly to the commodity flows themselves. This would mean that the same equations are in the end generated, but that the parameter representing the emission factor would be indepdendent of the unit. THat is, there would be a parameter $p^{EmissionFactor}_{GreenhouseGas,Coal}$ representing the emission factor, which is independent of the unit. In that case, the units should only specify a capture rate parameter if needed for a specific type of emission (default 0\%).

The generic equation would then become:
\begin{align}
&\vFlow[\commodity,\unit,out,\timestep] = \sum_{InputCommodity \in InputCommodities} p^{EmissionFactor}_{\commodity,InputCommodity} p^{CaptureRate}_{\unit,\commodity} \vFlow[InputCommodity,\unit,in,\timestep] \nonumber \\
&  \forall \commodity \in EmissionCommodities, \units, \timesteps
\end{align}

The advantage is that the user does not need to define the relationship between an input commodity and an emission commodity for every unit and that EmissionFactor is more easy to interpret than RatioOutputInputFlow. Instead, the user only needs to specify an emission factor for each commodity. At this point, I don't think there are disadvantages to this approach?}


\subsubsection{Minimum operating point}
%%%%%%%%%%%%%%%%%%%%%%%%%%%
Minimum operating point constraints can be introduced via the parameter $\pMinimumOperatingPoint$, which induces Eq.~\eqref{eq:minimumoperatingpoint}. This parameter needs to be specified for a specific commodity/commodity group. This specification is needed since it can be different than the capacity definining commodity group (Cdcg). For instance, for the coal-fired power plant considered here, the minimum operating point should be based on the ElectricalPower commodity and not on the sum of the ElectricalPower and the UpwardReserves commodity.


\subsubsection{Minimum up and down times}
%%%%%%%%%%%%%%%%%%%%%%%%%%%
Minimum up and down time constraints are induced by specifying the corresponding parameters ($\pMinimumUpTime$ and $\pMinimumDownTime$).


\subsubsection{Ramping constraints}
%%%%%%%%%%%%%%%%%%%%%%%%%%%
In this example, we assume a ramp rate which is constant for a given online unit, regardless of its current operating point. Depending on the stereotype, Eqs.~\eqref{eq:updwardrampingconstraintwithoutcommitmentvariables}-\eqref{eq:downdwardrampingconstraintwithoutcommitmentvariables} or Eqs.~\eqref{eq:updwardrampingconstraintwithcommitmentvariables}-\eqref{eq:downdwardrampingconstraintwithcommitmentvariables} will be generated.

Note that the ramping parameter again needs to be defined for a certain unit and commodity groups!




%%%%%%%%%%%%%%%%%%%%%%%%%%%%%%%%%%%%%%%%%%%%%%%%%%%%%
\subsection{Back-pressure CHP CCGT}
%%%%%%%%%%%%%%%%%%%%%%%%%%%%%%%%%%%%%%%%%%%%%%%%%%%%%
TO DO



%%%%%%%%%%%%%%%%%%%%%%%%%%%%%%%%%%%%%%%%%%%%%%%%%%%%%
\subsection{Extraction-condensing CHP CCGT}
%%%%%%%%%%%%%%%%%%%%%%%%%%%%%%%%%%%%%%%%%%%%%%%%%%%%%
TO DO




















% \bibliographystyle{abbrv}
% \bibliography{mybibfile}



%%%%%%%%%%%%%%%%%%%%%%%%%%%%%%%%%%%%%
%%%%%%%%%%%%%%%%%%%%%%%%%%%%%%%%%%%%%

% \bibliographystyle{abbrv}
% \bibliography{mybibfile}

\end{document}